\documentclass[11pt,a4paper,titlepage]{report}


% Document settings

\title{OSP Portfolio \\ Team $\langle$sql injection$\rangle$}

\author{
  Neil Ang\\
  \texttt{s3251533}
  \and
  ``Alfred" Yang Yuan\\
  \texttt{s3363619}
  \and
  Val Lyashov\\
  \texttt{s3366222}
}

\date{Semester 2, 2013}


% Change section numbering
%\renewcommand\thesection{\Roman{section}}
%\renewcommand\thesubsection{\Alph{subsection}}
\renewcommand\thesection{\arabic{section}}
\renewcommand\thesubsection{\thesection.\arabic{subsection}}


% Enable smart quotes
\usepackage [english]{babel}
\usepackage [autostyle]{csquotes}
\MakeOuterQuote{"}

% Alias pi name
\usepackage{xspace}
\newcommand{\rpi}{\textit{Raspberry Pi\textsuperscript{\textregistered}}}
\newcommand{\rpis}{\textit{Raspberry Pi\textsuperscript{\textregistered}s}}

% Side by side graphics
\usepackage{graphicx}
\usepackage{caption}
\usepackage{subcaption}

% Switch to biblatex
\usepackage{biblatex}
\bibliography{computer-vision}
\bibliography{audio}
\bibliography{servo}

% Add the bib to the toc
\DefineBibliographyStrings{english}{
  bibliography = {Bibliography},
}

% Better table height
\usepackage{tabu}

% The appendix
\usepackage{appendix}

% Code highlighting
\usepackage{listings}
\lstset{basicstyle=\ttfamily}

% For marking what's left to do
\usepackage{color}


\begin{document}


\maketitle

\pagebreak
\tableofcontents
\thispagestyle{empty}
\pagebreak

\section{Introduction}

\textcolor{red}{Can we use same as before?}


\section{Goals and Objectives}
\section{Feedback and Self-Reflection}
\subsection{Self Assessment}
\subsection{Summary of Prototype Demonstration}
\subsection{Peer Review Summary}

\textcolor{red}{This is waiting on the lab instructor to give us back our peer-review sheets.}


\subsection{Self Reflection / Lessons Learned}

\textcolor{red}{Individual responses here...}


\subsection{Description of how each learning objective is addressed}
\section{Assumptions and Dependencies}
\section{General Constraints}
\section{Development Methodology}

\textcolor{red}{Reinforce how we met learning objective 1 here.}


\subsection{Programming languages}

\textcolor{red}{Python for prototyping, C/C++ for performance.}


\subsection{Development tools}
\subsection{Collaboration tools}

\textcolor{red}{Git/GitHub}


\section{Difficulties Encountered}
\section{Architecture}
\subsection{System Design including configuration}
\subsection{Data Design}
\subsection{Program Design}
\subsubsection{Software Design}

\subsubsection{Source code or patches for all original work.}


Source code for simulating face detection. \textcolor{red}{TODO: move this under testing heading?}

\begin{lstlisting}[frame=single]
#include <iostream>
#include <cstdlib>
#include <unistd.h>
#include <stdio.h>
#include <string.h>

#define NA_DEFAULT_WAIT 3
#define NA_MIN_TOLERANCE 2
#define NA_DEFAULT_TOLERANCE 5

#define MAX(X,Y) ((X) > (Y) ? (X) : (Y))

int rand_num()
{
    return (rand() % 200 + 1) - 100;
}

int rand_with_tolerance(int t, int l)
{
    setvbuf(stdout, NULL, _IONBF, 0);
    int i;
    do {
        i = rand_num();
    } while (i > l+t || i < l-t);
    return i;
}

int main(int argc, const char * argv[])
{
    int x = 0;
    int y = 0;

    int t = NA_DEFAULT_TOLERANCE;
    int s = NA_DEFAULT_WAIT;

    for (int i = 1; i < argc; i++) {
        int v = i+1 <= argc;
        if (strcmp(argv[i], "-t") == 0 && v) {
            t = std::stoi(argv[i+1]);
            t = MAX(t, NA_MIN_TOLERANCE);
        }
        else if (strcmp(argv[i], "-s") == 0 && v) {
            s = std::stoi(argv[i+1]);
            s = MAX(s, 0);
        }
        else if (strcmp(argv[i], "-h") == 0) {
            std::cout << "Usage: -t <tolerance>
             -s <seconds>" << std::endl;
            return EXIT_SUCCESS;
        }

    }

    while (true) {
        x = rand_with_tolerance(t, x);
        y = rand_with_tolerance(t, y);
        printf("%d,%d\n", x, y);
        sleep(s);
    }

    return EXIT_SUCCESS;
}
\end{lstlisting}



\section{Testing Issues}
\subsection{Testing Done}
\subsection{Performance Bounds}
\subsection{Performance Experiments}

\section{Roles and Responsibilities}

As outlined in our original project specification, there was overlap in our teams individual strengths so the roles were divided with some shared responsibilities. Although at different levels, we wanted all three group members to be involved with experimenting with the hardware, writing software and documenting. The cross-over in roles made the progress slightly slower, but allowed each of us to trial something new.

\subsection{Val Lyashov}

\textcolor{red}{Add table about hardware solution}

\subsection{"Alfred" Yang Yuan}

\textcolor{red}{Add table about software solution}

\subsection{Neil Ang}

\textcolor{red}{Add table about whatever is left}


\section{Breakdown of Work Done by Team Member}

Each team member kept a detailed log of their activities. Below is a summary of what they achieved each week.

\subsection{Val Lyashov}

\begin{description}

  \item[Week 2] \hfill \\
      Formed team with Neil and discussed ideas for the project.
  \item[Week 3] \hfill \\
      Voted for a team restructure.
  \item[Week 4] \hfill \\
      Acquired new team member.
  \item[Week 5] \hfill \\
      Demonstrated cross-compile milestone in lab.
  \item[Week 6] \hfill \\
      ...
  \item[Week 7] \hfill \\
      ...
  \item[Week 8] \hfill \\
      Took a trip to Bunnings with Neil to build the servo mount.
  \item[Week 9] \hfill \\
      ...
  \item[Week 10] \hfill \\
      Presented project to lab group. Started work on final project portfolio submission.
  \item[Week 11] \hfill \\
      Worked on final project portfolio submission.
  \item[Week 12] \hfill \\
      Continued work on project portfolio submission.

\end{description}


\subsection{"Alfred" Yang Yuan}

\begin{description}

  \item[Week 4] \hfill \\
      Joined the team.
  \item[Week 5] \hfill \\
      Demonstrated cross-compile milestone in lab.
  \item[Week 6] \hfill \\
      ...
  \item[Week 7] \hfill \\
      ...
  \item[Week 8] \hfill \\
      ...
  \item[Week 9] \hfill \\
      ...
  \item[Week 10] \hfill \\
      Presented project to lab group. Started work on final project portfolio submission.
  \item[Week 11] \hfill \\
      Worked on final project portfolio submission.
  \item[Week 12] \hfill \\
      Continued work on project portfolio submission.

\end{description}

\subsection{Neil Ang}
See appendix for daily log of tasks.
\begin{description}

  \item[Week 2] \hfill \\
      Formed team with Val and discussed ideas for the project. Purchased \rpi. Completed cross-compile milestone. Installed Arch linux on \rpi. Started research into computer vision libraries.
  \item[Week 3] \hfill \\
      Purchased a RPi camera board. Installed Raspbian on \rpi. Researched depth sensing on the device. Prototyped face detection code on MBP with OpenCV. Acquired a web camera and tested on the device. Voted for a team restructure.
  \item[Week 4] \hfill \\
      Acquired new team member. Continued experimenting with OpenCV to improve performance running on the device. Started work on project specification.
  \item[Week 5] \hfill \\
      Demonstrated cross-compile milestone in lab. Continued work on project specification. Wrote test program to simulate face detection for Val.
  \item[Week 6] \hfill \\
      Finished the project specification. Ported the PIR motion sensing Python code to C.
  \item[Week 7] \hfill \\
      Researched IPC techniques. Acquired the servos and started work on porting code to C++.
  \item[Week 8] \hfill \\
      Solved issues with servo powers. Finished work on C++ wrapper for servos. Got face detection based movement working. Took a trip to Bunnings with Val to build the servo mount.
  \item[Week 9] \hfill \\
      Debugged issues with the servo wrapper when running on the Pi. Wrote simple daemon for automatically starting the server/client code. Wrote presentation and speech for next milestone.
  \item[Week 10] \hfill \\
      Presented project to lab group. Started work on final project portfolio submission.
  \item[Week 11] \hfill \\
      Worked on final project portfolio submission.
  \item[Week 12] \hfill \\
      Continued work on project portfolio submission.

\end{description}


\section{Summary and Conclusions}
\section{References}



\begin{appendices}

\chapter{Gantt Chart}

\includegraphics[width=\textwidth]{graphs/gantt-chart.pdf}


\chapter{Project Specification }

\textcolor{red}{Add project spec}



\end{appendices}

\nocite{*}
\printbibliography[heading=bibintoc]


\end{document}