\documentclass[11pt,a4paper,titlepage]{report}


% Document settings

\title{OSP Project Specification \\ Team $\langle$sql injection$\rangle$}

\author{
  Neil Ang\\
  \texttt{s3251533}
  \and
  ``Alfred" Yang Yuan\\
  \texttt{s3363619}
  \and
  Val Lyashov\\
  \texttt{s3366222}
}

\date{Semester 2, 2013}


% Change section numbering
%\renewcommand\thesection{\Roman{section}}
%\renewcommand\thesubsection{\Alph{subsection}}
\renewcommand\thesection{\arabic{section}}
\renewcommand\thesubsection{\thesection.\arabic{subsection}}


% Enable smart quotes
\usepackage [english]{babel}
\usepackage [autostyle]{csquotes}
\MakeOuterQuote{"}


% Set watermark while in development
\usepackage{draftwatermark}
\SetWatermarkText{DRAFT}
\SetWatermarkScale{5}
\SetWatermarkColor[gray]{0.95}


% Alias pi name
\usepackage{xspace}
\newcommand{\rpi}{\textit{Raspberry Pi\textsuperscript{\textregistered}}}
\newcommand{\rpis}{\textit{Raspberry Pi\textsuperscript{\textregistered}s}}


\begin{document}


\maketitle

\pagebreak
\tableofcontents
\thispagestyle{empty}
\pagebreak

\section{Introduction}

Our project will attempt to modernize the phonograph, by using the \rpi\xspace for subject tracking, digital recording and audio processing. A directional microphone will installed on a custom mount that can be programmatically moved in an X and Y direction. An additional camera will be used to detect the subject (e.g. a human face), and adjust the position of the microphone accordingly. The secondary of task of the project is to process the audio recordings through an Internet based voice-to-text service.

The final solution will be a relatively cheap, portable and smart recording device suitable for use in a lecture theatre or classroom. It would be ideal for assiting students in note taking, or for lecturers teaching in learning spaces that havn't been outfitted with full featured recording systems (such as the \textit{Lectopia}\footnote{http://www.rmit.edu.au/teaching/technology/lecturecapture} or \textit{Echo360}\footnote{http://www.rmit.edu.au/teaching/technology/echo360} systems used at \textit{RMIT}).


\section{System Overview}

This project will use a lot of hardware. Like heaps.

\begin{center}
\begin{tabular}{ |l|c|p{6cm}| }
    \hline
    
    Hardware & Quantity & Purpose \\ \hline
    
    \rpi & 2 & Due to the inherent complexity of audio and video processing, the solution workload will be split across two \rpis. The first will handle the computer vision and controlling the servos. The second is for processing audio and data storage.\\ \hline

    Camera & 1 & Either a RPi Camera board or USB web cam will be used for subject tracking through computer vision. \\ \hline
    
    Servos & 2 & Two servos will be used to control the microphone mount. One servo will maneuver the mount on an X axis, while the other is used for the Y axis. \\ \hline
        
    USB servo controller & 1 & This will be used to improve the movement accuracy of the servos \\ \hline
    
    USB sound card & 1 & The \rpi\xspace has no in-built audio input jack, so an external sound card will be used to facilitate connecting the microphone. \\ \hline
    
    
    Shotgun microphone & 1 & This microphone will give us targeted audio for recording voice and minimising background noise. \\ \hline
    
    
    Motion sensor & 1 & To conserve battery before and after a recording session, a low-power motion sensor will be used to detect the presence of a subject in the room. The solution will return to "standby mode" if no movement is detected.\\ \hline

    Power supply & 2 & To take advantage of the lightweight nature of the \rpi\xspace and components, an external battery pack will be used to make the solution portable.\\ \hline


    Wireless card & 1 & The voice-to-text processing will be performed remotely, so an unobtrusive internet connection will be required.\\ \hline
    
\end{tabular}
\end{center}

\section{Design Considerations}

...Power consumption and "standby mode".

\subsection{Goals and Objectives}

...

\subsection{Assumptions and Dependencies}

...

\subsection{General Constraints}

...

\subsection{Development Methodology}

...

\section{Architecture}

...

\subsection{System Design}

...

\subsection{Data Design}

...

\subsection{Program Design}

...

\subsubsection{Detailed Module Design}

...


\section{Testing Issues}

...

\subsection{Types of Testing to be conducted}

...

\subsection{Performance Bounds}

...




\section{Roles and Responsibilities}

...

\section{Biography}

...



\end{document}